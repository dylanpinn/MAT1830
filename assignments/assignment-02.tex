\documentclass[11pt]{article}
\usepackage{enumitem}
\usepackage{amsmath}
\begin{document}
\title{MAT1830 --- Assignment 2}
\author{Dylan Pinn --- 24160547}
\maketitle

\section*{Question 1}
Use truth tables to determine whether $(a \land \neg b) \to (b \lor c)$ is logically equivalent to $\neg a \lor b \lor c$.

\begin{tabular}{|c|c|c|c|c|c|c|}
\hline
$a$ & $b$ & $c$ & $a \land \neg $ & $b \lor c$ & $(a \land \neg b) \to (b \lor c)$ & $\neg a \lor b \lor c$ \\ \hline
T & T & T & F & T & T & T \\ \hline
T & T & F & F & T & T & T \\ \hline
T & F & T & T & T & T & T \\ \hline
T & F & F & T & T & T & F \\ \hline
F & T & T & F & T & T & T \\ \hline
F & T & F & F & F & T & T \\ \hline
F & F & T & F & T & T & T \\ \hline
F & F & F & F & F & F & T \\ \hline
\end{tabular}

Because they have different values they are not equivalent.

\break

\section*{Question 2}
Use laws of logic to show that $\neg (p \lor \neg (p \to q)) \lor p$ is a tautology.

\begin{equation}
	\neg (p \lor \neg (p \to q)) \lor p
\end{equation}
\begin{equation} \label{eq:1}
	\neg (p \lor \neg (\neg p \lor q)) \lor p
\end{equation}
Equation \ref{eq:1} uses Implication law.
\begin{equation} \label{eq:2}
	\neg (p \lor \neg (\neg p) \land \neg q) \lor p)
\end{equation}
Equation \ref{eq:2} uses De Morgan's law.
\begin{equation} \label{eq:3}
	\neg (p \lor p \land \neg q) \lor p	
\end{equation}
Equation \ref{eq:3} uses Double Negation law.
\begin{equation} \label{eq:4}
	\neg (p \land \neg q) \lor p
\end{equation}
Equation \ref{eq:4} uses Idempotent law.
\begin{equation} \label{eq:5}
	\neg p \land \neg (\neg q) \lor p
\end{equation}
Equation \ref{eq:5} uses De Morgan's law.
\begin{equation} \label{eq:6}
	\neg  p \lor q \lor p
\end{equation}
Equation \ref{eq:6} uses Double Negation law.
\begin{equation} \label{eq:7}
	T \lor q
\end{equation}
Equation \ref{eq:7} uses Inverse law.
\begin{equation} \label{eq:8}
	T
\end{equation}
Equation \ref{eq:8} uses Annihilation law.

This shows that it is a tautology.

\break

\section*{Question 3}
For each of the following statements, write down the statement's contrapositive and then write down the statement's negation.

\begin{enumerate}[label=(\roman*)]
\item "If the function is differentiable, then it is continuous."

	\textbf{Contrapositive}
	
	"If it isn't continuous, then the function is not differentiable."
	
	\textbf{Negation}
	
	"There exists a continuous function that is not differentiable."

\item "If a weak key was used, then the encryption wasn't secure."

	\textbf{Contrapositive}
	
	"If the encryption was secure, then a weak key wasn't used."
	
	\textbf{Negation}
	
	"The encryption was secure and a weak key was used."
	
\end{enumerate}
\break

\section*{Question 4}

Pei Ann has been dealt two cards from a standard 52 card deck. She holds one in her left hand and one in her right.  

Let $p$ be the proposition "The card in Pei Ann's left hand is an ace".

Let $q$ be the proposition "The card in Pei Ann's right hand is an ace".

Let $r$ be the proposition "The card in Pei Ann's left hand is a club".

Let $s$ be the proposition "The card in Pei Ann's right hand is a club".

Write down propositions (using just $p$, $q$, $r$, $s$ and logical connectives) corresponding to the following statements.

\begin{enumerate}[label=(\roman*)]

\item Neither of Pei Ann's cards is an ace.

$$\neg p \land \neg q$$

\item If Pei Ann has the ace of clubs in her right hand, then she doesn't have a club in her left hand

$$(q \land s) \to \neg r $$

\item Pei Ann has the ace of clubs and another club.

$$((p \land r) \lor (q \land s)) \land (r \lor s) $$

\end{enumerate}

\end{document}
