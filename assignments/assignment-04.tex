\documentclass[11pt]{article}
\usepackage{enumitem}
\usepackage{amssymb}
\usepackage{amsmath}
\begin{document}
\title{MAT1830 --- Assignment 3}
\author{Dylan Pinn --- 24160547}
\maketitle

\section*{Question 1}
Let $a_1,a_2,a_3,\dots$ be the sequence of integers defined by
$$a_1=3,a_2=9,\text{ and } a_i=12a_{i-1}-9a_{i-2} \text{ for each integer } i \geq 3$$
Prove by strong induction that $3^n$ divides $a_n$ for all integers $n \geq1$.

\textbf{Proof:} Let $P(n)$ be the statement "$3^n$ divides $a_n$".

\textbf{Base Steps:}
\begin{align*}
	a_1 &= 3 = 3^1 \times 1 \\
	a_2 &= 9 = 3^2 \times 1
\end{align*}

As $3^n$ divides both $P(1)$ and $P(2)$ are true.

\textbf{Induction Step:}
Suppose that $P(1),P(2),\dots,P(k)$ are true for some integer $k < n$. We want to prove that $P(k)$ is true when $k = n$.

\begin{align*}
	P(n) &= 12a_{n-1} - 9a_{n-2} \text{ (from above) } \\
	P(n-1) &= 3^{n-1}b \text{ (for some integer b)} \\
	P(n-2) &= 3^{n-2}c \text{ (for some integer c)} \\
	P(n) &= 12(3^{n-1}b) - 9(3^{n-2}c) \text{ (substituting)} \\
	&= 3^n(\frac{12b}{3^1} - \frac{9b}{3^2}) \\
	&= 3^n(4a - b)
\end{align*}

As $a$ and $b$ are integers then $4a-b$ is an integer and therefore $3^n(4a-b)$ is divisible by $3^n$.

\break
\section*{Question 2}
Let $R,S$ and $T$ be sets defined as follows.

\begin{align*}
	R &= \{ x: x \in \mathbb{Z} \text{ and either } x \leq -2 \text{ or } x \geq 5 \} \\
	S &= \{ -3,-2,4,5,6 \} \\
	T &= \{ x: x \in \mathbb{Z} \text{ and } x \geq 2 \}
\end{align*}

Find the following.

\begin{enumerate}[label=(\roman*)]
	\item $R \cap S$
	$$= \{ -3,-2,5,6 \}$$
	\item $R - T$
	$$= \{x: x \in \mathbb{Z} \text{ and } x \leq -2 \} $$
	\item $R \triangle S$
	$$= \{x: x \in \mathbb{Z} \text{ and } x < -3 \text{ or } x > 6 \}$$
	\item $\mathcal{P} (R) \cap \{ \{-6,-3,5\},\{4,5\},\{7\},\{\},\{-3,1,7\}\}$
	$$= \{\{7\},\{\}\}$$
	\item $|\mathcal{P}(\mathcal{P}(S \cap T))|$
	\begin{align*}
		U &= S \cap T = \{4,5,6\} \\
		Y &= |\mathcal{P}(U)| = 2^3 = 8 \\
		|\mathcal{P}(Y)| &= 2^8 = 256
	\end{align*}
	Therefore $|\mathcal{P}(\mathcal{P}(S \cap T))| = 256$
\end{enumerate}

\break
\section*{Question 3}

\begin{enumerate}[label=(\roman*)]
	\item Is $(A\cap B) \times C = (A \times C) \cap (B \times C)$ true for all sets $A,B$ and $C$? If, so prove it. If not, give an example of sets $A,B$ and $C$ for which it is false.

	$(A\cap B) \times C = (A \times C) \cap (B \times C)$

	\begin{enumerate}[label=\arabic*]
		\item So, $(x,y) \in (C\times A)$ and $(x,y)\in(C\times B)$,
		same as

		$(x,y)\in(C\times A) \cap (C \times B)$, so

		$C \times (A \times B) \subseteq (C \times A)\cap (C \times B)$

		\item So, $(x,y) \in (C \times A) \cap (C \times B)$ and need to show that

		$(x,y) \in C \times (A \cap B)$

		We know that $(x,y)\in (C \times A)$ and $(x,y)\in (C \times B)$.

		Definition of cartesian product gives us $x \in A$ and $x \in B$ and $x \in C$.

		Therefore, $y \in A \cap C$ and $(x,y)\in C \times (A \cap B)$, so

		$(C \times A)\cap(C \times B) \subseteq C \times (A \times B)$
	\end{enumerate}

	As both equations in 1 and 2 are subsets of each other they are equal.

	\item Is $\mathcal{P}(A)\cup \mathcal{P}(B) = \mathcal{P}(A \cup B)$ true for all sets $A$ and $B$? If, so prove it. If not, give an example of sets $A$ and $B$ for which it is false.

	Not true. For example $A = \{1\}$ and $B=\{2,3\}$

	\begin{align*}
		\mathcal{P}(A) &= \{\emptyset,\{1\}\} \\
		\mathcal{P}(B) &= \{\emptyset,\{2\},\{3\},\{2,3\}\} \\
		\mathcal{P}(A) \cup \mathcal{P}(B) &= \{\emptyset,\{1\},\{2\},\{3\},\{2,3\}\} \\
		\mathcal{P}(A \cup B) &= \{\emptyset,\{1\},\{2\},\{3\},\{1,2\},\{1,3\},\{2,3\},\{1,2,3\}\}
	\end{align*}

	Therefore $\mathcal{P}(A)\cup \mathcal{P}(B) \not= \mathcal{P}(A \cup B)$.
\end{enumerate}

\end{document}
