\documentclass[11pt]{article}
\usepackage{enumitem}
\usepackage{amsmath}
\usepackage{booktabs}

\begin{document}
\title{MAT1830 --- Assignment 9}
\author{Dylan Pinn --- 24160547}
\maketitle

\section*{Question 1}

A biased coin flips heads with probability $\frac{4}{9}$ and tails with
probability $\frac{5}{9}$. The coin is flipped 90 times. What is the probability
that heads is flipped exactly 40 times?

\begin{align*}
  \Pr(H) &= \frac{4}{9} \\
  \Pr(T) &= \frac{5}{9} \\
  \Pr(X = k) &= \binom{n}{k} p^k {(1-p)}^{n-k} \text{ for } k \in \{ 0,
    \dots, n \} \\
  n &= 90 \\
  k &= 40 \\
  p &= \frac{4}{9} \\
  \Pr(X = 40) &= \binom{90}{40} {\frac{4}{9}}^{40} {(1-\frac{4}{9})}^{90-40} \\
  &= \frac{90!}{40!50!} \times \frac{4^{40}}{9^{40}} \times
  \frac{5^{50}}{9^{50}} \\
  &\approx 0.08
\end{align*}

\break{}

\section*{Question 2}

Cars pass through a road junction according to a Poisson distribution. An
average of 5 cars per minute pass through the junction.

\begin{enumerate}[label= (\alph*)]
  \item What is the probability that exactly one car passes through the
    junction in a certain minute?

  $\lambda = 5$ and using the Poisson distribution formula gives;

  \begin{align*}
    \Pr(X=k) &= \frac{{\lambda}^k {e}^{- \lambda}}{k!} \\
    \Pr(X=1) &= \frac{5^1 e^{-5}}{1!} \\
    &= \frac{5}{e^5} \\
    &\approx 0.03
  \end{align*}

  \item Write down the expected number of cars to pass through the junction in
    four minutes.

    20 cars in 4 minutes.

  \item What is the probability that exactly 10 cars pass through the junction
    in a certain four minute period?

    $\lambda = 20$ from part (b) and using the Poisson distribution gives;

    \begin{align*}
      \Pr(X=10) &= \frac{20^{10} e^{-20}}{10!} \\
      &= \frac{1600000000}{567 e^{20}} \\
      &\approx 0.0058
    \end{align*}
\end{enumerate}

\break{}
\section*{Question 3}

A random variable $Y$ can only take values in $ \{ -5, 0, 5 \}$. The expected
value of $Y$ is 0 and its variance is 24. Find the probability distribution of
$Y$.

\begin{center}
  \begin{tabular}{c c c c}
    \toprule
    $y$ & -5 & 0 & 5 \\
    $\Pr(Y=y)$ & $a$ & $b$ & $c$ \\
    \bottomrule
  \end{tabular}
\end{center}

We know that $a + b + c = 1$ and $-5 + 5c = 0$ which gives $a = c$.

\[ a{(-5)}^2 + b{(0)}^2 + c{(5)}^2 = 24 \]
\[ 25a + 25c = 24 \]
\[ 25(a + c) = 24 \]
\[ a + c = \frac{24}{25} \]
as $a = c$ \[ 2a = \frac{24}{25} \]
\[ a = \frac{24}{50} = \frac{12}{25} \]

We can substitute this into the original formula to give:

\[ a + b + c = 1 \]
\[ b = 1 - a - c \]
\[ = 1 - \frac{12}{24} - \frac{12}{24} \]
\[ = \frac{1}{25} \]

\begin{center}
  \begin{tabular}{c c c c}
    \toprule
    $y$ & -5 & 0 & 5 \\
    $\Pr(Y=y)$ & $\frac{12}{24}$ & $\frac{1}{24}$ & $\frac{12}{24}$ \\
    \bottomrule
  \end{tabular}
\end{center}

\break{}
\section*{Question 4}

Write down the first five values $(r_0, \dots, r_4 \text{ and } s_0, \dots,
s_4)$ of each of the following recursive sequences.

\begin{enumerate}[label = (\alph*)]
  \item $r_0 = 2$, $r_n = {(r_{n-1})}^2 - 3n + 1$ for all integers $n \geq 1$.

    \begin{align*}
      r_0 &= 2 \\
      r_1 &= {(r_0)}^2 - 3 \times 1 + 1 \\
      &= 2^2 - 3 + 1 \\
      &= 0 \\
      r_2 &= {(r_1)}^2 - 3 \times 2 + 1 \\
      &= 0^2 - 6 + 1 \\
      &= -7 \\
      r_3 &= {(r_2)}^2 - 3 \times 3 + 1 \\
      &= {(-7)}^2 - 9 + 1 \\
      &= 39 \\
      r_4 &= {(r_3)}^2 - 3 \times 4 + 1 \\
      &= 39^2 - 12 + 1 \\
      &= 1510
    \end{align*}

  \item $s_0 = 2$, $s_n = {(s_{n-1})}^2 + {(s_{n-2})}^2 + \cdots + {(s_0)}^2$
  for all integers $n \geq 1$.

    \begin{align*}
      s_0 &= 2 \\
      s_1 &= 2^2 \\
          &= 4 \\
      s_2 &= 4^2 + 2^2 \\
          &= 20 \\
      s_3 &= 20^2 + 4^2 + 2^2 \\
          &= 420 \\
      s_4 &= 420^2 + 20^2 + 4^2 + 2^2 \\
          &= 176820
    \end{align*}
\end{enumerate}

\end{document}
