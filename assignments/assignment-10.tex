\documentclass[11pt]{article}
\usepackage{enumitem}
\usepackage{amsmath}
\usepackage{booktabs}

\begin{document}
\title{MAT1830 --- Assignment 10}
\author{Dylan Pinn --- 24160547}
\maketitle

\section*{Question 1}
Rewrite the following expressions without using $\Sigma$ or $\Pi$.

\begin{enumerate}[label= (\alph*)]
  \item $\sum_{i=2}^{6} \frac{2}{6i - 7}$

  \item $\prod_{i=5}^{8} ({(z + 2i)}^{i} - i - 1)$

\end{enumerate}

\section*{Question 2}
Rewrite the following expressions using $\Sigma$ or $\Pi$ notation.

\begin{enumerate}[label= (\alph*)]
  \item $x(x+1)(x+4)(x+9)(x+16) \dots (x+400)$

  \item $\frac{1}{6^4} + \frac{1}{9^5} + \frac{1}{12^6} + \frac{1}{15^7} + \dots
    + \frac{1}{33^{13}}$

\end{enumerate}

\section*{Question 3}
Call a string of letters ``legal'' if it can be produced by concatenating
(running together) copies of the strings `a', `bb' and `cc'. For example, `abba'
is legal because it can be produced by concatenating `a', `bb' and `a', but
`ccca' is not legal.

For each integer $n \geq 1$, let $t_n$ be the number of legal strings wiht $n$
letters. For example, $t_1 = 1$ (`$a$' is the only legal string) and $t_2 = 3$
(`aa', `bb' and `cc' are the legal strings).

\begin{enumerate}[label= (\alph*)]
  \item Write down $t_3$ and a list of all the legal strings of length 3.

  \item Write down $t_4$ and a list of all the legal strings of length 4.

  \item Find a recurrence for $t_n$ that holds for all $n \geq 3$. Explain why
    your recurrence gives $t_n$.

\end{enumerate}

\section*{Question 4}
Draw a simple graphs with the following properties or explain why they do not
exist.

\begin{enumerate}[label= (\alph*)]
  \item The list of verticies is: $P$, $Q$, $R$, $S$, $T$ and teh list of edges
    is $PQ$, $PS$, $QR$, $RS$, $RT$.

  \item The graph has 10 verticies and 47 edges.

  \item The graph has 7 verticies and 6 edges and is connected.

  \item The graph has 7 verticies and 11 edges and its verticies can be divided
    into two sets in such a way that no edge joins two verticies in the same
    set.

\end{enumerate}

\end{document}
