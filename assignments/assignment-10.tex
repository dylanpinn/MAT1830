\documentclass[11pt]{article}
\usepackage{enumitem}
\usepackage{amsmath}
\usepackage{booktabs}

\begin{document}
\title{MAT1830 --- Assignment 10}
\author{Dylan Pinn --- 24160547}
\maketitle

\section*{Question 1}
Rewrite the following expressions without using $\Sigma$ or $\Pi$.

\begin{enumerate}[label= (\alph*)]
  \item $\sum_{i=2}^{6} \frac{2}{6i - 7}$

  \begin{align*}
    &= \frac{2}{6 \times 2 - 7} + \frac{2}{6 \times 3 - 7} + \frac{2}{6 \times
    4 - 7} \\
    &+ \frac{2}{6 \times 5 - 7} + \frac{2}{6 \times 6 - 7} + \frac{2}{6 \times
    7 - 7} \\
    &= \frac{2}{5} + \frac{2}{11} + \frac{2}{17} + \frac{2}{23} + \frac{2}{29} +
    \frac{2}{35}
  \end{align*}

  \item $\prod_{i=5}^{8} ({(z + 2i)}^{i} - i - 1)$

    \begin{align*}
      &= ({(z + 2 \times 5)}^{5} - 5 - 1) \times ({(z + 2 \times 6)}^{6} - 6 -
      1) \\
      &\times ({(z + 2 \times 7)}^{7} - 7 - 1) \times ({(z + 2 \times 8)}^{8} - 8
      - 1) \\
      &\times ({(z + 2 \times 9)}^{9} - 9 - 1) \times ({(z + 2 \times 10)}^{10}
      - 10 - 1) \\
      &\times ({(z + 2 \times 11)}^{11} - 11 - 1) \times ({(z + 2 \times
      12)}^{12} - 12 - 1) \\
      &= ({(z + 10)}^{5} - 6) \times ({(z + 12)}^{6} - 7) \times ({(z + 14)}^{7}
      - 8) \\
      &\times ({(z + 16)}^{8} - 9) \times ({(z + 18)}^{9} - 10) \times ({(z +
      20)}^{10} - 11) \\
      &\times ({(z + 22)}^{11} - 12) \times ({(z + 24)}^{12} - 13) \\
    \end{align*}

\end{enumerate}

\section*{Question 2}
Rewrite the following expressions using $\Sigma$ or $\Pi$ notation.

\begin{enumerate}[label= (\alph*)]
  \item $x(x+1)(x+4)(x+9)(x+16) \dots (x+400)$

    \[ \prod_{i=0}^{21} (x + i^2) \]

  \item $\frac{1}{6^4} + \frac{1}{9^5} + \frac{1}{12^6} + \frac{1}{15^7} + \dots
    + \frac{1}{33^{13}}$

    \[ \sum_{i=4}^{10} \frac{1}{{(3i - 6)}^{i}} \]

\end{enumerate}

\section*{Question 3}
Call a string of letters ``legal'' if it can be produced by concatenating
(running together) copies of the strings `a', `bb' and `cc'. For example,
`abba' is legal because it can be produced by concatenating `a', `bb' and `a',
but `ccca' is not legal.

For each integer $n \geq 1$, let $t_n$ be the number of legal strings with $n$
letters. For example, $t_1 = 1$ (`$a$' is the only legal string) and $t_2 = 3$
(`aa', `bb' and `cc' are the legal strings).

\begin{enumerate}[label= (\alph*)]
  \item Write down $t_3$ and a list of all the legal strings of length 3.

    \begin{itemize}
      \item $aaa$
      \item $abb$
      \item $acc$
      \item $bba$
      \item $cca$
    \end{itemize}

    \[ t_3 = 5 \]

  \item Write down $t_4$ and a list of all the legal strings of length 4.

    \begin{itemize}
      \item $aaaa$
      \item $abba$
      \item $aabb$
      \item $aacc$
      \item $acca$
      \item $bbaa$
      \item $bbbb$
      \item $bbcc$
      \item $ccaa$
      \item $cccc$
      \item $ccbb$
    \end{itemize}

    \[ t_4 = 11 \]

  \item Find a recurrence for $t_n$ that holds for all $n \geq 3$. Explain why
    your recurrence gives $t_n$.

\end{enumerate}

\section*{Question 4}
Draw simple graphs with the following properties or explain why they do not
exist.

\begin{enumerate}[label= (\alph*)]
  \item The list of vertices is: $P$, $Q$, $R$, $S$, $T$ and the list of edges
    is $PQ$, $PS$, $QR$, $RS$, $RT$.

  SEE ATTACHED

  \item The graph has 10 vertices and 47 edges.

  Does not exist. Maximum number of edges possible for a simple graph of $n$
    verticies is $\binom{n}{2}$.

  \[ \binom{10}{2} = 45 \]

  So a simple graph of 10 vertices and 47 edges is not possible.

  \item The graph has 7 vertices and 6 edges and is connected.

  Does not exist. To traverse between 7 verticies you need at least 7 edges.

  \item The graph has 7 vertices and 11 edges and its vertices can be divided
  into two sets in such a way that no edge joins two vertices in the same set.

\end{enumerate}
\end{document}
